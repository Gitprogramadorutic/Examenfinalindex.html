\documentclass{article}
\usepackage[utf8]{inputenc}
\usepackage[spanich]{}

\title{2A-1 ACTIVIDAD- Tabular Archivo}
\author{Antonia Dominguez}
\date{21 de Junio 2021}

\begin{document}

\maketitle

\section{Ejercicio}

Para obtener una simple tabla sin lineas\\

\begin{tabular}{l c r}

 1 & 2 & 3 \\
 4 & 5 & 6 \\
 7 & 8 & 9 \\   
\end{tabular}

\section{Ejercicio}

Para obtener esta tabla en la añadimos algunas lineas verticales:\\

\begin{tabular}{l|c||r}
  1 & 2 & 3 \\
  4 & 5 & 6 \\
  7 & 8 & 9 \\
\end{tabular}

\section{Ejercicio}
Ahora con lineas Horizontales: Superior e Inferior \\

\begin{tabular}{l|c||r|}
\hline
   1 & 2 & 3 \\
   4 & 5 & 6 \\
   7 & 8 & 9 \\
 \hline
\end{tabular}

\section{Ejercicio}
Ahora añadir lineas centradas entre todas las filas (usamos el entorno center)\\
\begin{center}
\begin{tabular}{l|c||r|}
\hline
   10 & 22 & 33 \\ \hline
   44 & 55 & 66 \\ \hline
   77 & 88 & 99 \\ 
 \hline
\end{tabular}
\end{center}


\section{Ejercicio}

Departamento de la República del Paraguay.\\

\begin{tabular}{|l|l|l|r}
\hline
    1 & \textbf{Dto Alto Paraguay} & Alto Paraguayo \\ \hline
    2 & \textbf{Dpto Alto Parana} & Alto Paranaense \\ \hline
    3 & \textbf{Dpto Guaira } & Guaireño \\ \hline
    4 & \textbf{Dpto Misiones} & Misionero \\ \hline
    5 & \textbf{Dpto Caaguazu} & Caaguazeño \\ \hline
    6 & \textbf{Dpto Caazapa} & Caazapeño \\ \hline
    7 & \textbf{Dpto Concepción} & Concepcionero \\ \hline
    8 & \textbf{Dpto San Pedro} & San Pedrano \\ \hline
    9 & \textbf{Dpto Itapua} & Itapuense \\ \hline
    10 & \textbf{Dpto Paraguari} & Paraguariense \\ \hline
\end{tabular} 
\end{document}


    










